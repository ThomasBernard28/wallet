\documentclass{article}
\usepackage{float}
\usepackage{svg}
\usepackage{amsmath}
\usepackage[utf8]{inputenc}
\usepackage[french]{babel}
\usepackage{graphicx}
\usepackage[T1]{fontenc}
\usepackage{hyperref}
\usepackage{changepage}
\hypersetup{
    colorlinks,
    citecolor=black,
    filecolor=black,
    linkcolor=black,
    urlcolor=black
}
\graphicspath{ressources/}
\usepackage{tikz}
\usetikzlibrary{shapes,positioning}

\usepackage{blindtext}

\usepackage{subfiles} % Best loaded last in the preamble

\title{Titre}
\author{ }
\date{ }

\begin{document}
\begin{titlepage}
    \begin{center}
        
        {\Large Université de Mons}\\[1ex]
        {\Large Faculté Des Sciences}\\[1ex]
        {\Large Département d'Informatique}\\[1ex]
        
        \newcommand{\HRule}{\rule{\linewidth}{0.3mm}}
        % Title
        \HRule \\[0.3cm]
        { \LARGE \bfseries Projet de modélisation logicielle \\[0.3cm]}
        { \LARGE \bfseries Rapport de modélisation  \\[0.1cm]} % Commenter si pas besoin
        \HRule \\[1.5cm]
        
        % Author and supervisor
        \begin{minipage}[t]{0.45\textwidth}
            \begin{flushleft} \large
                \emph{Professeur:}\\
                Tom \textsc{Mens}\\
                \emph{Assistants:}\\
                Jeremy \textsc{Dubrulle}\\
                Gauvin \textsc{Devillez}\\
                Sébastien \textsc{Bonte}\\
            \end{flushleft}
        \end{minipage}
        \begin{minipage}[t]{0.45\textwidth}
            \begin{flushright} \large
                \emph{Auteurs:} \\
                Pignozi \textsc{Agbenda} \\
                Thomas \textsc{Bernard} \\
                Théo \textsc{Godin} \\
                Ugo \textsc{Proietti}\\
            \end{flushright}
        \end{minipage}\\[2ex]
        
        \vfill
        
        % Bottom of the page
        \begin{center}
            \begin{tabular}[t]{c c c}
                \includegraphics[height=1.5cm]{ressources/logoumons.jpg} &
                \includegraphics[height=1.5cm]{ressources/logofs.jpg} &
            \end{tabular}
        \end{center}~\\
        
        {\large Année académique 2021-2022}
        
    \end{center}
\end{titlepage}

\title{Projet de Modélisation Logicielle}
\author{AGBENDA Christian, BERNARD Thomas, GODIN Theo et PROIETTI Ugo}
\maketitle
\tableofcontents
\newpage

\pagenumbering{arabic}

\section{Introduction}
		Ce rapport recense tous les diagrammes UML composant nos applications de base ainsi que les 4 extensions de notre groupe. Ce rapport contient également le 
		diagramme d'API ainsi que les diagrammes d'entité-relation général et ses versions étendues pour chacune de nos extensions.  
		Le projet est composé de 2 applications : L'application 1 qui est une application de gestions de portfeuilles d'un point de vue client et l'application 2 qui 
		est l'application de gestion du point de vue institutionnel. Ces deux applications ont une modélisation assez semblable en terme d'achitecture mais diffèrent
		l'une de l'autre du point de vue de leurs fonctionnalités.
		\bigskip

		Notre phase de modélisation a parfois été semée d'embuches. Tantôt dans la compréhension des consignes tantôt dans la découverte de certains concepts informatiques
		qui nous étaient encore inconnus jusque là. Nous avons toutefois su faire nos recherche afin d'amener notre compréhension à un niveau suffisant pour nous permettre
		de manipuler non sans difficulté et maladresse ces nouveaux concepts.

		\bigskip

		Cette modélisation nous a permis de mieux nous projeter dans ce que sera l'implémentation du projet.
\section{Applications de bases}

	\subsection{Application1 : Gestion de portefeuilles financiers}

		\subsubsection{Vue d'ensemble}
			L'application 1 est l'application qui est destinée aux clients. Il s'agit de l'application sur
			laquelle nous avons commencé à travailler et donc celle qui a subi le plus de modifications au total.
			Il est possible que certains concepts soient modifiés lors de l'implémentation notamment ceux qui 
			n'ont pas été approfondis au maximum. Cette application sera étendue par l'entièreté du groupe.
		\newpage
		\subsubsection{Diagramme de cas d'utilisation}
				
				\subfile{sections/usecase1}
		\subsubsection{Interaction Overview Diagram}
		
				\subfile{sections/rapportInteractionOverviewApp1}
		
		\subsubsection{Diagramme de classes}
				\subfile{sections/class1}
		\newpage
		\subsubsection{Diagrammes de séquences}

                \subfile{sections/sequences1}
		
	\subsection{Application 2 : Application de gestion pour insitutions financières}

		\subsubsection{Vue d'ensemble}
			L'application 2 est l'application destinée aux institutions. Cette application leur permet de gérer 
			leurs clients ainsi que les produits de ceux-ci. La conception de cette application est basée sur celle 
			de l'application 1. En effet l'application 2 reprend certains des aspects de l'applications 1. Tout 
			en étant plus restrictive. En effet, l'application 2 n'a pas d'accès à la notion de portfeuille. 
			Toutefois, elle possède des outils de gestion qui ne font pas partie de l'applciation 1.
		\subsubsection{Diagramme de cas d'utilisation}
		\subfile{sections/usecase2}
		\newpage
		\subsubsection{Interaction Overview Diagram}

				\subfile{sections/rapportInteractionOverviewApp2}
		\newpage
		\subsubsection{Diagramme de classes}
				\subfile{sections/class2}
		\newpage
		\subsubsection{Diagramme de séquences}
				\subfile{sections/sequences2}
	\subsection{Serveur}
		\subsubsection{Vue d'ensemble}
        \subfile{sections/vueEnsembleServeur}

		
		\subsubsection{Diagramme d'entité-relation}
		
		\subfile{sections/erd}
	\newpage	
		\subsubsection{API}
		
		\subfile{sections/API}
   
    \subsection{Interface Graphique}
		\subsubsection{Application 1}
	        \subfile{sections/rapportGui}
		\newpage
		\subsubsection{Application 2}
		     \subfile{sections/rapportGui2}

\section{Extensions}
	
	\subsection{Extension 1, Gestion des cartes - Godin Théo: }
	\subfile{sections/extensionTheo}
	\subsection{Extension 2, Gestion des devises et virements internationnaux - Proietti Ugo : }
        \subfile{sections/extensionUgo}
	
	\subsection{Extension 5, Gestion des contrats d'assurance - Bernard Thomas : }
		\subfile{sections/extensionThomas}
	\subsection{Extension 6, Paiement et gestion des fraudes - Agbenda Pignozi : }
		

\end{document}
