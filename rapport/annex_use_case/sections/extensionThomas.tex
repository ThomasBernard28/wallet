\documentclass[../annexe.tex]{subfiles}
\graphicspath{{\subfix{ressources}}}

\begin{document}

\subsection{Demander un devis pour un type d'assurance}

\paragraph{Acteurs :} Le client et le serveur 


\paragraph{Descrition :} Permet au client de demander un devis pour un certain type d'assurance.


\paragraph{Préconditions :} Le client se trouve sur l'écran lié à la demande de devis.


\paragraph{Fréquence :} De temps en temps (lorsque le client désire un devis).



\paragraph{Parcours de base :}

	\begin{enumerate}
	
		\item Le client clique sur l'option de demande d'un devis pour une assurance.
		\item Le client choisit l'assurance propre à l'institution pour laquelle il désire avoir un devis.
		\item L'application envoie les informations au serveur.
		\item Le serveur renvoie le devis.
		\item Le client consulte le devis.
	\end{enumerate}
	
\paragraph{Postconditions :} La GUI affiche le devis que le client a demandé.

\newpage

\subsection{Obtenir des informations sur les différents types 
d'assurances}

\paragraph{Acteurs :} Le client et le serveur 

\paragraph{Description :} Permet au client d'obtenir des informations sur toutes les assurances que propose l'insitution financière à laquelle est lié le portefeuille sélectionné.

\paragraph{Préconditions :} Le client doit être connecté, avoir sélectionné un  de ses portefeuilles être entré dans la gestion des produits financiers.

\paragraph{Fréquence :} De temps en temps.

\paragraph{Parcours de base :}

	\begin{enumerate}
		\item Le client clique sur la fenêtre liée aux informations concernant les assurances.
		\item L'application envoie la demande d'informations au serveur avec l'institution pour laquelle elles sont demandées.
		\item Le serveur renvoie la liste des assurances de l'institution ainsi que tous les détails les concernants.
		\item Le client consulte les informations.
	\end{enumerate}
	
\paragraph{Postconditions :} La GUI affiche la liste des assurances pour l'institution sélectionnée ainsi que toutes les informations qui en découlent.

\newpage

\subsection{Accéder à la liste des assurances :}

\paragraph{Acteurs :} Le client et le serveur.

\paragraph{Description :} Un client accède à la liste des assurances auxquelles il a soucrit dans l'institution qui correspond au portefeuille sélectionné.

\paragraph{Préconditions :} Le client doit avoir choisi un de ses portefeuille auquel il voulait accéder et doit avoir choisi de gérer ses produits financiers.

\paragraph{Fréquence :} Assez souvent

\paragraph{Parcours de base :} 

	\begin{enumerate}
		\item Le client clique sur l'accès à ses assurances.
		\item L'application envoie au serveur une demande de récupération des assurances pour le client en particulier.
		\item Le serveur renvoie les diverses assurances auxquelles le client a soucrit s'il en a. 
		Si pas le serveur ne renvoie rien.
		\item Le client consulte ses assurances.
	\end{enumerate}

\paragraph{Postconditions :} La GUI affiche la liste des assurances du client et la possibilité d'en rajouter.

\paragraph{Points d'extensions :} 
\begin{enumerate}
	\item Résilier une assurance
	\item Souscrire à une assurance
\end{enumerate}

\newpage

\subsection{Gérer les paramètres :}

\paragraph{Acteurs :} Le client et le serveur.

\paragraph{Description :} Le client accède aux paramètres d'une des assurances auxquelles il a souscrit. 

\paragraph{Préconditions :} Le client doit avoir sélectionné une des assurances présentes dans sa liste d'assurances.

\paragraph{Fréquence :} Peu souvent

\paragraph{Parcours de base :}

	\begin{enumerate}
		\item Le client sélectionne une de ses assurances.
		\item L'application envoie une demande au serveur pour recevoir les informations liées à cette assurance.
		\item Le serveur renvoie les informations de cette assurance.
		\item Le client clique sur le bouton des paramètres pour cette assurance.
		\item L'application affiche l'onglet des paramètres pour assurance.
		\item Le client effectue des modifications.
		\item L'application envoie les modifications au serveur.
		\item Le serveur effectue les modifications dans la base de donnée.
		\item Le serveur envoie les nouvelles informations à l'application.
		\item L'application affiche les nouvelles informations.
		\item Le client quitte les paramètres.
	\end{enumerate}

\paragraph{Postconditions :} La GUI affiche l'assurance avec les nouveaux paramètres qui lui ont été appliqués.

\paragraph{Parcours alternatif :} Le client n'effectue pas de modifications et quitte la fenêtre.

\newpage

\subsection{Visualiser l'historique :}

\paragraph{Acteurs :} Le client et le serveur 

\paragraph{Description :} Le client accède à l'historique concernant ses assurances. Il peut y voir ses souscriptions, ses résiliations, ses paiements et ses retraits.

\paragraph{Préconditions :} Le client doit se trouver dans la liste des assurances.

\paragraph{Fréquence :} Peu souvent 

\paragraph{Parcours de base :} 

	\begin{enumerate}
		\item Le client clique sur le bouton d'affichage de l'historique dans la fenêtre des assurances.
		\item L'application envoie la demande d'historique spécifique au client au serveur.
		\item Le serveur renvoie les informations liées à l'historique.
		\item L'application affiche l'historique des assurances du client
		\item Le client consulte son historique 
	\end{enumerate}

\paragraph{Postconditions :} La GUI affiche l'historique des assurances du client.

\newpage

\subsection{Souscrire à une assurance :}

\paragraph{Acteurs :}

\paragraph{Description :}

\paragraph{Préconditions :}

\paragraph{Fréquence :}

\paragraph{Parcours de base :}

\paragraph{Postconditions :}

\newpage

\subsection{Résiliez une assurance :}

\paragraph{Acteurs :}

\paragraph{Description :}

\paragraph{Préconditions :}

\paragraph{Fréquence :}

\paragraph{Parcours de base :}

\paragraph{Postconditions :}

\newpage

\subsection{Effectuer une transaction :}

\paragraph{Acteurs :}

\paragraph{Description :}

\paragraph{Préconditions :}

\paragraph{Fréquence :}

\paragraph{Parcours de base :}

\paragraph{Postconditions :}

\newpage

\subsection{Payer la primer :}

\paragraph{Acteurs :}

\paragraph{Description :}

\paragraph{Préconditions :}

\paragraph{Fréquence :}

\paragraph{Parcours de base :}

\paragraph{Postconditions :}

\newpage

\subsection{Verser/retirer de l'argent d'une assurance :}

\paragraph{Acteurs :}

\paragraph{Description :}

\paragraph{Préconditions :}

\paragraph{Fréquence :}

\paragraph{Parcours de base :}

\paragraph{Postconditions :}




\end{document}