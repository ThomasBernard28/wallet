\documentclass[../annexe.tex]{subfiles}
\graphicspath{{\subfix{ressources}}}

\begin{document}


\subsection{Use-case Application 1}


    \subsubsection{Demande de paiement par code QR} 

        \paragraph{Acteurs: Client d’une institution}
        \paragraph{Description: Le client génère un code QR qui, une fois scanné, permettra d’effectuer une transaction}
        \paragraph{Précondition: Le clien est connecté à l’application}
        \paragraph{Fréquence: Occasionnelle}
        \paragraph{Parcours de base:} 
        \begin{enumerate}
        \item  Le client clique sur le bouton QR code.
        \item  Le client choisit l’option «  create payment request ».
        \item  Le client entre le numéro du compte bancaire sur le quel sera versé l’argent.
        \item  Le client entre le montant de la transaction.
        \item  Le client entre un message qui sera affiché quand la transaction sera effectuée.
        \item  Le client clique sur bouton « generate ».
        \item  L’application génère un code QR et le stock localement. 
        \end{enumerate}
            \paragraph{Parcours alternatif:\  } 
        \paragraph{Postcondition: Le code QR est généré. }

\newpage    


    \subsubsection{Payement par code QR }

        \paragraph{Acteurs: Client d’une institution et le serveur}
        \paragraph{Description: Le client effectue une transaction en scannant un code QR.}
        \paragraph{Précondition: Le clien est connecté à l’application.}
        \paragraph{Fréquence: Occasionnelle}
        \paragraph{Parcours de base:}
        \begin{enumerate}
            \item  Le client clique sur le bouton QR code.
            \item  Le client choisit l’option « load payment request ».
            \item  Le client entre le numéro du compte bancaire sur le quel l’argent sera débité.
            \item  Le client clique sur le bouton «load».
            \item  Le client choisit le code QR à scanner.
            \item  Le client confirme le scan.
            \item  Le l’application envoie la requête au serveur.
            \item  Le serveur effectue la requête. 
        \end{enumerate}
            \paragraph{Parcours alternatif:} 
        \begin{itemize}
            \item[7.b] Une erreur de connexion a lieu et la requête a échoué, le client est averti,
            il est redirigé vers le menu de payement par code QR
        \end{itemize} 
        \paragraph{Postcondition: La transaction est effectuée.}
\newpage
  
        \subsubsection{Effecteur un payement avec/sans contact }
  
        \paragraph{Acteurs: Client d’une institution et le serveur}
        \paragraph{Description: Le client effectue un virement rapide avec/sans contact}
        \paragraph{Précondition: Le client est connecté à l’application}
        \paragraph{Fréquence: Souvent}
        \paragraph{Parcours de base: }
        \begin{enumerate}
            \item  Le client sélectionne l’option de payement rapide.
            \item  Le client entre le compte qui sera débité.
            \item  Le client sélectionne la transaction qu’il souhaite effectuer.
            \item  Le client choisit le payement avec contact.
            \item  Le client entre le code PIN du compte bancaire.
            \item  Le client confirme la transaction.
            \item  Le l’application envoie la requête au serveur.
            \item  Le serveur effectue la requête.
        \end{enumerate}
            \paragraph{Parcours alternatif:} 
        \begin{itemize}
            \item[4.b] Le client choisit le payement sans contacte, la transaction est effectué
            \item[5.b] Le client se trompe de code PIN, il ne lui reste plus que 2 chances pour entrer le bon code PIN
            \item[8.b] Une erreur de connexion a lieu et la requête a échoué, le client est averti, il est redirigé vers le menu de payement rapide
        \end{itemize} 
        \paragraph{Postcondition: Le virement est effectué.}
\newpage
  
        \subsubsection{Gérer les limites des payement avec/sans contact}
  
        \paragraph{Acteurs: Employé d’une institution financière et le serveur}
        \paragraph{Description: L’employé de l’institution financière peux changer le plafond limite (quotidien, hebdomadaire, mensuel) pour les payements avec ou sans contact}
        \paragraph{Précondition: L’em­ployé est connecté} 
        \paragraph{Fréquence: Occasionnelle}
        \paragraph{Parcours: de base} 
        \begin{enumerate}
            \item  L’employé clique sur le bouton pour définir les plafonds.
            \item  L’employé entre les différents plafonds.
            \item  L’employé confirme les plafonds.
            \item  L’application envoie la requête au serveur.
            \item  Le serveur effectue la requête.
        \end{enumerate}
        \paragraph{Parcours alternatif:} 
        \begin{itemize}
            \item[2.b] L’employé coche l’option pour ne pas mettre de plafonds, aucun plafond n'est appliqué.
            \item[5.b] Une erreur de connexion a lieu et la requête a échoué, l’employé est averti, il est redirigé vers le menu de définition de plafonds.
        \end{itemize} 
            \paragraph{Postcondition: Les plafonds sont définis.}
    

\end{document}