\documentclass[../rapport.tex]{subfiles}
\graphicspath{{\subfix{ressources/}}}


\begin{document}

	\subsection{Extension 5 BERNARD Thomas : Gestion des Assurances}
		
		\paragraph{Statut global de l'extension}
			Suite au manque de temps rencontré par notre groupe, l'extension n'est pas
			fonctionnelle, certaines scènes et controllers de GUI ont été implémentés du côté
			du frontend mais ce n'est pas complet. Le jar du frontend lié à l'extension n'est
			pas la denrière version du frontend mais celà ne pose pas de problème par rapport
			à l'extension.

			\medskip

			C'est du côté du backend que l'extension a été la plus implémentée. En effet,
			comme je travaillais sur le backend il m'a été plus facile d'implémenter 
			une partie de mon extension.

		\subsubsection{Frontend}
			
			\paragraph{Interface graphique}
			Au niveau de l'interface graphique les scènes permettant de visualiser la 
			liste des assurances a été implémentée ainsi que celle en permettant l'ajout.
			
			\medskip

			Les scènes ne chargent malheureusement pas le contenu correctement et ne sont donc
			pas fonctionnelles car j'ai préféré me focaliser sur la partie backend qui me 
			semblait plus importante afin de pouvoir par après gérer la partie frontend.
			
			\medskip

			La scène des assurances est accessible depuis la liste des produits via le bouton
			situé au bas de la scène.

		
		\subsubsection{Backend}

			\paragraph{Base de donnée et API :}
			Seule une partie de l'etension a été implémentée. En effet, j'ai préféré me focaliser
			sur un type d'assurance afin d'en développer un maximum de fonctionnalités, plutôt
			que de faire tous les types mais sans fonctionnalité.

			\medskip

			Toutefois, la structure actuelle de la BDD permet l'ajout facile de nouveaux types
			d'assurances. On retrouve une table générale \textit{INSURANCES} pouvant contenir
			tous les types d'assurances et puis une table plus spécifique pour les autres types
			d'assurances. Dans notre cas c'est la table \textit{PENSION\_SAVINGS} qui correspond
			à une épargne pension. Celle-ci est fonctionnelle d'un point de vue backend.
			On peut la créer, y ajouter de l'argent à épargner, le pourcentage de retour 
			est calculé en fonction du montant ajouté et des plafonds sont implémentés.

			\medskip

			Les transactions fonctionnent de la même manière que pour celles entre comptes
			bancaires. Il y une table \textit{PENSION\_BALANCES} qui ressemble à la table
			\textit{CASH\_BALANCES} et qui correspond au sol de l'épargne pension. 
			
			\medskip

			De plus, toute transacton depuis un compte vers l'assurance est également répertoriée
			dans l'historique des transactions du compte afin de garder une trace sur les 
			transactions effectuées.

			\medskip

			Finalement, il manque la fonctionnalité permettant de rétribuer le souscripteur 
			de l'assurance au bout de l'année car celle-ci n'a pas eu le temps d'être implémentée
			mais pourrait être ajoutée facilement grâce au Scheduler déjà implémenté.


			\bigskip

			\paragraph{Conclusion : } Pour cette extension il manque principalement de types
			d'assurances variés ainsi que d'une réelle implémentation au niveau du frontend.
			Bien que les requêtes HTTP soit fonctionnelles, leur interprétation par le côté 
			client n'est quant à elle pas fonctionnelle.

			\medskip

			Il manque également la partie institution de l'extension qui était quant à elle
			minime et se concentrait plutôt sur une gestion des devis qui pourrait être
			implémentée facilement grâce à la table \textit{INSURANCE\_INFOS} contenant les
			diverses informations des assurances en fonction de leur type.
			

			\newpage

	\subsection{Extension 1 GODIN Theo : Gestion des cartes}
		
		\paragraph{Statut global de l'extension}
		Suite au retard global accumulé sur le projet, seuls les classes de l'application et une scène de l'interface ont été implémentés.\\
		La scène de connexion via une carte aurait permis à l'utilisateur de s'authentifier en simulant un lecteur de carte. Mais la partie backend de mon extension n'est pas du tout implémentée. 

\end{document}
