\documentclass[../rapport.tex]{subfiles}

\begin{document}

\subsection{Design et inspiration:}
    
L'idée derrière l'interface graphique de l'application 2 était la même que pour l'application 1 avec un petit changement qui est le fait 
qu'on a voulu implémenter des fonctionnalités qui se déclencheraient avec un click gauche.


\subsection{Outils:}

    Ayant décidé de faire les différentes scènes en fxml et de gérer les interactions avec des contrôleurs, nous avons beaucoup utilisé 
    SceneBuilder qui a grandement facilité la création des différents fichiers fxml. Le fait de pouvoir visualiser directement à quoi ressemblait
    une scène sans avoir besoin de compiler et d'exécuter le fichier mais aussi le fait d'avoir une interdace facile d'utilisation qui nous 
    permettait de facilement ajouter des éléments aux scènes nous a vraiment aidés. Le seul problème que nous avons eu avec SceneBuilder est que
    le rendu qu'il affiche ne correspondait pas toujours à ce qu'on avait une fois la scène affichée sur nos écrans.


\subsection{Différences avec les schémas:}

    Les différences avec le modèle fournist au premier quadrimestre sont le fait qu'on a ajouté des scènes mais ceux décrivent déjà schématiser 
    ont été gardés et sont identiques hormis quelques modifications de l'apparence.


\subsection{Manques à corriger:}

    Comme décrit au point précédent, le plus gros problème qu'on a eu avec l'application 2 était l'implémentation de click gauche qui
    ferais apparaitre un sous-menu. Nous avons d'abord essayé de créer les sous-menus et de les afficher en récupérant les coordonnées de 
    l'endroit où se trouvait la souris au moment du click et d'y faire apparaitre le sous-menu. Malheureusement on n'arrivait pas à 
    soit faire apparaitre le menu au bon endroit, soit le faire disparaitre une fois un autre click effectuer donc on se retrouvait 
    avec un écran rempli de sous-menus. La solution qu'on a trouvé bien trop tard pour l'implémenter fu l'utilisation de la classe
    ContextMenu qui étaient une class spécialement dédié à la création de sous-menus. Mais pour cause notre mauvaise gestion de temps
    nous n'avons pu explorer cette piste plus en profondeur ce qui rend le GUI de l'application 2 incomplète.

    \newpage
\end{document}