\documentclass[../rapport.tex]{subfiles}

\begin{document}



\subsubsection{Ajouter un client}

\textbf{Acteurs :} \\
Employé d'une institution et serveur. \\

\textbf{Description :} \\
Ajoute un nouvau client. \\

\textbf{Précondition :} \\
L'employé de l;institution est connecté. \\

\textbf{Fréquence :} \\
Souvent. \\

\textbf{Parcours de base :} \\
\begin{enumerate}
    \item L’employé de l’institution appuie sur le bouton d’ajout de client
    \item L’employé de l’institution entre les donnée du client (nom, prénom, numéro	de registre national)
    \item L’employé de l’institution entre les donnée du produit financier auquel le client souhaite souscrire
    \item L’application envoie un message de confirmation
    \item L’employé de l’institution accepte
    \item L’application envoie la requête d’ajout au serveur
\end{enumerate}
\bigskip

\textbf{Parcours alternatif :}
\begin{enumerate}
    \item L'employé de l'institution refuse, le message se ferme et aucune opération n'est effectuée et l'employé de l'institution est renvoyé sur la fenêtre d'ajout de client
    \item Une erreur de connexion a lieu et la requête a échoué, l'employé de l'institution est averti et est renvoyé sur la fenêtre d'ajout de client
\end{enumerate}

\textbf{Postconditions :} \\
Le client est ajouté comme clinet de l'institution financière. \\



\subsubsection{Consulter la liste des clients et leurs produits financiers}

\textbf{Acteurs :} \\
L'employé de l'instiution et le serveur. \\

\textbf{Description :} \\
Permet a  l’employé de l’institution de pouvoir consulter la liste des clients de l’institution financière chez qui l’employé travail ainsi que leurs produits financier. \\

\textbf{Précondition :} \\
L’employé de l’institution est connecté. \\

\textbf{Parcours de base :} \\
\begin{enumerate}
    \item L’employé de l’institution appuie sur le bouton pour consulter la liste de clients ainsi que leurs produits financier
    \item L’application envoie une requête au serveur 
    \item Le serveur reçoit la requête
    \item Le serveur envoie les données nécessaire
\end{enumerate}
\bigskip

\textbf{Parcours alternatif :}
\begin{enumerate}
    \item Il y a un problème de connexion, l'employé de l'institution est averti et est renvoyé sur le menu de base
\end{enumerate}

\textbf{Postconditions :} \\
L'application affiche la liste de client et leurs produits financiers. \\



\subsubsection{Activer les virements}

\textbf{Acteurs :} \\
Employé d'une institution et serveur. \\

\textbf{Description :} \\
. \\

\textbf{Précondition :} \\
L'employé d'une institution est connecté et le client de l'institution a effectué la demande d'activation de virement bancaire. \\

\textbf{Fréquence :} \\
. \\

\textbf{Parcours de base :} \\
\begin{enumerate}
    \item L'employé d'une institution appuie sur le bouton pour consulter les demandes d'activation de virement
    \item L'employé d'une institution accepte la demande du client
    \item L'application envoie un message de confirmation
    \item L'empployé d'une institution confirme
    \item Les informations sont envoyées au serveur
    \item Le serveur reçoit les informations
\end{enumerate}
\bigskip

\textbf{Parcours alternatif :}
\begin{enumerate}
    \item L'employé d'une institution refuse et il est renvoyé sur la fenêtre contenant les demandes d'activation de virement.
    \item Il y a une problème de connexion, l'employé de l'institution est averti et est renvoyé sur la fenêtre contenant les demandes d'activation de virement
\end{enumerate}

\textbf{Postconditions :} \\
Les virements sont autorisés pour le client. \\



\subsubsection{Supprimer un client}

\textbf{Acteurs :} \\
Employé d'une institution et serveur. \\

\textbf{Description :} \\
Supprime définitivement un client de l'institution. \\

\textbf{Précondition :} \\
L’employé d’une institution est connecté, le client est client chez l’institution de l’employé. \\

\textbf{Fréquence :} \\
Occassionel. \\

\textbf{Parcours de base :} \\
\begin{enumerate}
    \item L’employé d’une institution  appuie sur le bouton de suppression
    \item L’employé d’une institution  entre les données du client à supprimer
    \item L’application affiche un message de confirmation 
    \item L’employé d’une institution accepte
    \item L’application envoie la requête de suppression au serveur
\end{enumerate}
\bigskip

\textbf{Parcours alternatif :}
\begin{enumerate}
    \item L’employé d’une institution effectue un clique droit sur un client puis l’application affiche des options supplémentaires, l’employé d’une institution sectionne la suppression du client, l’application affiche un message de confirmation, l’employé d’une institution confirme, la requête est envoyé au serveur
    \item L’employé d’une institution refuse, la fenêtre se ferme et aucune opération n’est effectué
    \item Une erreur de connexion a lieu et la requête a échoué, l’employé d’une institution et averti et retourne la liste des clients
\end{enumerate}

\textbf{Postconditions :} \\
Le client est supprimé et ne figure plus dans la liste des clients. \\



\subsubsection{Supprimer un produit financier}

\textbf{Acteurs :} \\
Employé d'une institution et serveur. \\

\textbf{Description :} \\
Supprime définitivement un prodiut financier d'un client. \\

\textbf{Précondition :} \\
L'employé d'une institution est connecté, le client est client chez l'institution de l'employé. \\

\textbf{Fréquence :} \\
Occassionel. \\

\textbf{Parcours de base :} \\
\begin{enumerate}
    \item L’employé d’une institution appuie sur le bouton de suppression
    \item L’employé d’une institution  entre les données du client et du produit à supprimer
    \item L’application affiche un message de confirmation
    \item L’employé d’une institution accepte
    \item L’application envoie la requête de suppression au serveur
\end{enumerate}
\bigskip

\textbf{Parcours alternatif :}
\begin{enumerate}
    \item L’employé d’une institution effectue un clique droit sur un client puis l’application affiche des options supplémentaires, l’employé d’une institution sectionne la suppression du produit, l’ employé sélectionne le produit à supprimer l’application affiche un message de confirmation, l’employé d’une institution confirme, la requête est envoyé au serveur 
    \item L’employé d’une institution refuse, la fenêtre se ferme et aucune opération n’est effectué
    \item Une erreur de connexion a lieu et la requête a échoué, l’employé d’une institution  est averti et retourne la liste des clients
\end{enumerate}

\textbf{Postconditions :} \\
le produit financier est supprimé et ne figure plus dans la liste du client qui le possédait. \\



\subsubsection{Ajouter un produit financier}

\textbf{Acteurs :} \\
Employé d'une institution et serveur. \\

\textbf{Description :} \\
Ajoute un produit financier à un client. \\

\textbf{Précondition :} \\
L’employé d’une institution est connecté le client possède un compte chez l’institution. \\

\textbf{Fréquence :} \\
Souvent. \\

\textbf{Parcours de base :} \\
\begin{enumerate}
    \item L’employé d’une institution appuie sur le bouton d’ajout
    \item L’employé d’une institution entre les données du client et du produit à ajouter
    \item L’application affiche un message de confirmation
    \item L’employé d’une institution accepte
    \item L’application envoie la requête de suppression au serveur
\end{enumerate}
\bigskip

\textbf{Parcours alternatif :}
\begin{enumerate}
    \item L’employé d’une institution effectue un clique droit sur un client puis l’application affiche des options supplémentaires, l’employé d’une institution sectionne l’ajout d’un produit, l’ employé entre les données du produit à ajouter l’application affiche un message de confirmation, l’employé d’une institution confirme, la requête est envoyé au serveur
    \item L’employé d’une institution refuse, la fenêtre se ferme et aucune opération n’est effectué
    \item Une erreur de connexion a lieu et la requête a échoué, l’employé d’une institution est averti et retourne la liste des clients
\end{enumerate}

\textbf{Postconditions :} \\
Le produit financier est ajouté et figure dans la liste du client qui le possède. \\



\subsubsection{Trier les produits financiers}

\textbf{Acteurs :} \\
Employé d'une institution. \\

\textbf{Description :} \\
L'employé d'une institution trie la liste des produits financiers. \\

\textbf{Précondition :} \\
L'employé d'une institution est connecté. \\

\textbf{Fréquence :} \\
Souvent. \\

\textbf{Parcours de base :} \\
\begin{enumerate}
    \item L’ employé d’une institution sélectionne en fonction de quoi il souhaite trier les produis financier
    \item L’application trie les produits financier
\end{enumerate}
\bigskip

\textbf{Postconditions :} \\
La liste est triée. \\

\newpage
\end{document}
